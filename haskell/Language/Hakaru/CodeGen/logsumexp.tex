\documentclass{article}
\usepackage[margin=0.8in]{geometry}
\usepackage{amsmath}
\usepackage{tikz}
\usepackage{tikz-qtree}
\usepackage{verbatim}

% \usepackage{natbib}
% \twocolumn


\title{Adding Probabilities in the Hakaru to C Compiler (HKC)}
\author{Zach Sullivan}
\date{July 30, 2016}

\begin{document}
\maketitle

\section*{Probabilities in Hakaru}

Because Hakaru is a probabilistic language, we provide a type just for
probabilities. The type {\tt prob} has extra safety from underflow. The main way
we accomplish this is by storing its value as double precision floating point
numbers in the log-domain. We can do basic arithmetic on our probabilities.

\section*{The ``LogSumExp Trick''}
Because our probability types are stored in the log-domain, we need to compute
\begin{displaymath}
{\rm LSE}(x) = \log\Bigg(\sum_{i=0}^{n}{e^{x_i}}\Bigg)
\end{displaymath}
often called ``LogSumExp.'' The advantage of storing probabilities in the
log-domain is lost if we simply remove them from the log-domain before doing
calculations on them. We would also add a bunch of {\tt log} and {\tt exp}
operations.

\begin{align*}
{\rm LSE^\prime}(x) &= \hat x + \log\Bigg(\sum_{i=0}^{n}{e^{x_i-\hat x}}\Bigg)\\
{\rm\bf where}&\ \hat x = {\rm max}(x)
\end{align*}

This trick preserves the value of the largest of the numbers being summed.


\subsection*{Max Comparison Tree}
LogSumExp safety creates a particular challenge when generating code for our
compiler. HKC compiles Hakaru to C, where we have no max function that works
on an arbitrary number of arguments. The solution is to create a tree of
comparisons using C's ternary conditional expression to find the maximum of $n$
number of arguments.

Here is an example of a max comparison tree when the length of array $x$ is
4.

\begin{center}
\Tree [ .{$x_0 > x_1$}
        [ .{$x_0 > x_2$}
          [ .{$x_0 > x_3$}
            % {$x_0$}
            % {$x_3$}
          ]
          [ .{$x_2 > x_3$}
            % {$x_2$}
            % {$x_3$}
          ]
        ]
        [ .{$x_1 > x_2$}
          [ .{$x_1 > x_3$}
            % {$x_1$}
            % {$x_3$}
          ]
          [ .{$x_2 > x_3$}
            % {$x_2$}
            % {$x_3$}
          ]
        ]
      ]
\end{center}

\subsection*{Code Generation}

After generating a max comparison tree, we can create leaves for our tree that
are different LogSumExp summations. This is rather trivial given a particular
max index.

% We use {\tt log1p} and {\tt expm1} to help prevent.

A Hakaru program that sums 4 probabilities will generate the following C
expression, where {\tt p\_a}, {\tt p\_b}, {\tt p\_c}, and {\tt p\_d} are
variables holding probabilities:

\begin{center}
\begin{verbatim}
p_a > p_b
  ? p_a > p_c
    ? p_a > p_d
      ? p_a + log1p(expm1(p_c - p_a) + (expm1(p_d - p_a) + expm1(p_b - p_a)) + 3)
      : p_d + log1p(expm1(p_b - p_d) + (expm1(p_c - p_d) + expm1(p_a - p_d)) + 3)
    : ( p_c > p_d
      ? p_c + log1p(expm1(p_b - p_c) + (expm1(p_d - p_c) + expm1(p_a - p_c)) + 3)
      : p_d + log1p(expm1(p_b - p_d) + (expm1(p_c - p_d) + expm1(p_a - p_d)) + 3))
  : (p_b > p_c
    ? p_b > p_d
      ? p_b + log1p(expm1(p_c - p_b) + (expm1(p_d - p_b) + expm1(p_a - p_b)) + 3)
      : p_d + log1p(expm1(p_b - p_d) + (expm1(p_c - p_d) + expm1(p_a - p_d)) + 3)
    : (p_c > p_d
      ? p_c + log1p(expm1(p_b - p_c) + (expm1(p_d - p_c) + expm1(p_a - p_c)) + 3)
      : p_d + log1p(expm1(p_b - p_d) + (expm1(p_c - p_d) + expm1(p_a - p_d)) + 3)));
\end{verbatim}
\end{center}

We can use {\tt log1p} because where $x_i$ is the maximum
$e^{x_i - \hat x} = 1$. We use {\tt log1p} and {\tt expm1} because they can be
more accurate for small values.


\subsection*{Future Improvements}

The LogSumExp code generation will be the same for each $n$ number of
arguments. If we have several LogSumExp operations of size $n$ in the same
program, then we will be generating the same code multiple times. Creating
different LogSumExp C functions for different numbers of arguments will reduce
code size.

Kahan summation is another improvement that keeps track of the accumulated
error in each addition, which would add more robustness to our operations.

% \bibliographystyle{te}
% \bibliography{research}

\end{document}
